% Chapter 1

\chapter{Counting} % Main chapter title

\label{Unit 3} % For referencing the chapter elsewhere, use \ref{Chapter1} 

%-------------------------------------------------------------------------------------------------------
% Define some commands to keep the formatting separated from the content 

%%%%%%%%%%%%%%%%%%%%%%%%%%%%%%%%%%%%%%%%
%%%%%%%%%%---------------SECTION START---------------%%%%%%%%%%
%%%%%%%%%%%%%%%%%%%%%%%%%%%%%%%%%%%%%%%%
\section{Lecture 4: Counting}

%-------------------------------------------------------------------------------------------------------
\subsection{Discrete uniform law}
\begin{outline}
\1 Assume $\Omega$ consists of $n$ equaly likely elements
\1 Assume $A$ consists of $k$ elements
\1 Then: $\color{red} P(A) = \frac{\text{number of elements of } A}{\text{number of elements of } \Omega} = \frac{k}{n}$
\1 Applications
  \2 Permutations, combinations
  \2 Partitions
  \2 Number of subsets
  \2 Binomial probabilities
\end{outline}

%-------------------------------------------------------------------------------------------------------
\subsection{Basic counting principle}

\noindent{\textbf{Example}}
\begin{outline}
\1 4 shirts, 3 ties, 2 jackets: number of possible attires?
  \2 $r$ selection stages
  \2 $n_i$ choices at stage $i$
\1 Number of choices is: $n_1 \cdot n_2 \cdot n_3 ... \cdot n_r$
\end{outline}

\noindent{\textbf{More examples}}
\begin{outline}
\1 Number of license plates with 2 letters followed by 3 digits:
  \2 $26 \cdot 26 \cdot 10 \cdot 10 \cdot 10$
\1 \textbf{Permutations:} Number of ways of ordering $n$ elements:
  \2 $n \cdot (n-1) \cdot (n-2) ... \cdot 1 = n!$
\1 Number of subsets of {1, ..., n}:
  \2 $2 \cdot 2 \cdot ... \cdot 2 = 2^n$
\end{outline}


%-------------------------------------------------------------------------------------------------------
\subsection{Combinations}

\begin{outline}
\1 Definition: $\color{red} \binom{n}{k}$: number of $k$-element subsets of a given $n$-element set $\color{red}=\frac{n!}{k!(n-k)!}$
\1 Two ways of constructing an \textbf{ordered} sequence of $k$ \textbf{distinct} items:
  \2 Choose the $k$ items one at a time
  \2 Choose $k$ items, then order them
\end{outline}

%-------------------------------------------------------------------------------------------------------
\subsection{Partitions}

\begin{outline}
\1 $n \geq 1$ distinct items; $r \geq 1$ persons, given $n_i$ items to person $i$
  \2 here $n_1, ..., n_r$ are given nonnegative integers
  \2 with $n_1 + ... n_r = n$
\1 Ordering $n$ items: $n!$
  \2 Deal $n_i$ to each persons $i$, and then order
\1 Number of partitions $\color{red} =\frac{n!}{n_1!n_2!...n_r!}$ (multinomial effect)
\end{outline}


%%%%%%%%%%---------------          END          ---------------%%%%%%%%%%


%%%%%%%%%%---------------          END          ---------------%%%%%%%%%%



