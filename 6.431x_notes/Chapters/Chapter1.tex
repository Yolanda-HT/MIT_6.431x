% Chapter 1

\chapter{Probability models and axioms} % Main chapter title

\label{Unit 1} % For referencing the chapter elsewhere, use \ref{Chapter1} 

%-------------------------------------------------------------------------------------------------------
% Define some commands to keep the formatting separated from the content 
\newcommand{\keyword}[1]{\textbf{#1}}
\newcommand{\tabhead}[1]{\textbf{#1}}
\newcommand{\code}[1]{\texttt{#1}}
\newcommand{\file}[1]{\texttt{\bfseries#1}}
\newcommand{\option}[1]{\texttt{\itshape#1}}

%%%%%%%%%%%%%%%%%%%%%%%%%%%%%%%%%%%%%%%%
%%%%%%%%%%---------------SECTION START---------------%%%%%%%%%%
%%%%%%%%%%%%%%%%%%%%%%%%%%%%%%%%%%%%%%%%
\section{Math Overview}

%-------------------------------------------------------------------------------------------------------
\subsection{Sets and De Morgan's Laws} 
\noindent{\textbf{\textcolor{brown}{Sets}}}
\begin{itemize}
\item A collection of distinct element
\item Can be finite or infinite
\end{itemize}

%-------------------------------------------------------------------------------------------------------
\noindent{\textbf{\textcolor{brown}{Unions and intersections}}
\begin{itemize}
\item $S \cup T: \quad x \in S \cup T \Leftrightarrow x \in S$ or $x \in T $
\item $S \cap T: \quad x \in S \cap T \Leftrightarrow x \in S $ and $x \in T$
\item $x \in \underset{n}{\cup} S_n \Leftrightarrow x \in S_n$, for  some $n$
\item $x \in \underset{n}{\cap} S_n \Leftrightarrow x \in S_n$, for all $n$
\end{itemize}

%-------------------------------------------------------------------------------------------------------
\noindent{\textbf{\textcolor{brown}{Set properties}}
\begin{itemize}
\item $S \cup T = T \cup S$
\item $S \cap (T \cup U) = (S \cap T) \cup (S \cap U) $
\item $(S^c)^c = S$
\item $S \cup \Omega = \Omega$
\item $S \cup (T \cup U) = (S \cup T) \cup U$
\item $S \cup (T \cap U) = (S \cup T) \cap (S \cup U)$
\item $S \cap S^c = \emptyset$
\item $S \cap \Omega = S$
\end{itemize}

%-------------------------------------------------------------------------------------------------------
\noindent{\textbf{\textcolor{brown}{De Morgan's laws}}
\begin{itemize}
\item $(\underset{n}{\cap} S_n)^c = \underset{n}{\cup} S_n^c$
\item $(\underset{n}{\cup} S_n)^c = \underset{n}{\cap} S_n^c$
\end{itemize}

%-------------------------------------------------------------------------------------------------------
\noindent{\textbf{\textcolor{brown}{Sequences and their limits}}

\begin{outline}
\1 \textbf{Definition of Sequence} 
  \2 function$ f: \mathbb{N} \rightarrow S, f(i) = a_i$
\1 \textbf{Convergence of Sequence} 
  \2 $a_i \underset{i \rightarrow \infty}{\rightarrow} a, \lim\limits_{i\to\infty} a_i = a$ 
  \2 For any $\epsilon > 0$, there exists $i_0$, such that if $i \geq i_0$, then $|a_i - a| < \epsilon$
\end{outline}


%-------------------------------------------------------------------------------------------------------
\subsection{Sequences and their limits}

\begin{outline}
 \1 If $a_i \geq a_{i+1}$, for all $i$, then either:
   \2 the sequence "converges to $\infty$"
   \2 the sequence converges to some real number a
 \1 If $|a_i - a| \leq b_i$, for all $i$, and $b_i \rightarrow 0$, then $a_i \rightarrow a$
 \1 Properties of convergent sequences
   \2 If $a_i \rightarrow a$ and $b_i \rightarrow b$, then
     \3 $a_i + b_i \rightarrow a + b$
     \3 $a_i b_i \rightarrow a b$
   \2 If $a_i \rightarrow a$ and $g$ is a continuous function, then
     \3 $g(a_i) \rightarrow g(a)$
\end{outline}

%-------------------------------------------------------------------------------------------------------
\subsection{Infinite series}

Provided limit exists: $\sum\limits_{i=1}^{\infty}a_i = \lim\limits_{n \rightarrow \infty} \sum\limits_{i=1}^{n} a_i$

\begin{outline}
\1 If $a_i \geq 0$: limit exists
\1 If term $a_i$ do not all have the same sign:
  \2 limit need not exist
  \2 limit may exist but be different if we sum in a different order
  \2 \textbf{Fact:} limit exists and independent of order of summation if $\sum\limits_{i=1}^{\infty} |a_i| < \infty$
\end{outline}

%-------------------------------------------------------------------------------------------------------
\subsection{Geometric series}

$\sum\limits_{i=0}^{\infty}\alpha^i = 1 + \alpha + \alpha^2 + ... + = \frac{1}{1-\alpha}$ $|\alpha| < 1$

%-------------------------------------------------------------------------------------------------------
\subsection{Sums with multiple indices}

$\sum\limits_{i\geq1, j\geq1} a_{ij}$
\begin{outline}
\1 If the sum converges, this double series will be well defined.
\1 If $\sum |a_{ij}| < \infty$, then order of summation does not matter.
\end{outline}


%-------------------------------------------------------------------------------------------------------
\subsection{Countable and uncountable sets}

\begin{outline}
\1 Countable: can put in 1-1 correspondence with positive integers
  \2 positive integers
  \2 integers
  \2 pairs of positive integers
  \2 rational number $q$, with $0 < q < 1$
    \3 $1/2, 1/2, 2/3, 1/4, 2/4, 3/4, 1/5, 2/5 ...$
\1 Uncountable: not countable
  \2 the interval $[0,1]$
  \2 the reals, the plane, ...
\1 The reals are uncountable
  \2 Cantor's diagonalization argument
\end{outline}


%%%%%%%%%%---------------          END          ---------------%%%%%%%%%%



%%%%%%%%%%%%%%%%%%%%%%%%%%%%%%%%%%%%%%%%
%%%%%%%%%%---------------SECTION START---------------%%%%%%%%%%
%%%%%%%%%%%%%%%%%%%%%%%%%%%%%%%%%%%%%%%%
\section{Lecture 1: Probability models and axioms}

%-------------------------------------------------------------------------------------------------------
\subsection{Sample space}
\begin{outline}
\1 Two steps:
  \2 Describe possible outcomes
  \2 Describe beliefs about likelihood of outcomes
\1 List (set) of possible outcomes, $\Omega$
  \2 Mutually exclusive
  \2 Collectively exhaustive
  \2 At the "right" granularity
\1 Examples
  \2 Discrete / finite
    \3 Two rolls of a tetrahedral die
    \3 Sequential description (decision tree)
  \2 Continuous
    \3 $(x,y)$ such that $0 \leq x, y \leq 1$
\end{outline}


%-------------------------------------------------------------------------------------------------------
\subsection{Probability laws}
\begin{outline}
\1 \textbf{\textcolor{brown}{Event:}} a subset of the sample space
  \2 Probability is assigned to events
\1 \textbf{\textcolor{brown}{Axioms:}}
  \2 Nonnegativity: $P(A) \geq 0$
  \2 Normalization: $P(\Omega) = 1$
  \2 (Finite) additivity: (to be strengthened later)
    \3 If $A \cap B = \emptyset$, then $P(A \cup B) = P(A) + P(B)$
\end{outline}

%-------------------------------------------------------------------------------------------------------
\subsection{Some simple consequences of the axioms}
\begin{outline}
\1 $P(A) \leq 1$
\1 $P(\emptyset) = 0$
\1 $P(A) + P(A^C)  = 1$
\1 $P(A \cup B \cup C) = P(A) + P(B) + P(C)$ and similarly for $k$ disjoint events $P({s_1, s_2, ... s_k}) = P(s_1) + ... + P(s_k)$
\1 $A \cup A^C = \Omega$
\1 $A \cap A^C = \emptyset$
\end{outline}

%-------------------------------------------------------------------------------------------------------
\subsection{More consequences of the axioms}
\begin{outline}
\1 If $A \subset B$, then $P(A) \leq P(B)$
\1 $P(A \cup B) = P(A) + P(B) - P(A \cap B)$
\1 $P(A \cup B) \leq P(A) + P(Bs)$
\1 $P(A \cup B \cup C) = P(A) + P(A^C \cap B) + P(A^C \cap B^C \cap C)$
\end{outline}

\textbf{Examples}
\begin{outline}
\1 \textbf{\textcolor{brown}{Discrete / finite example:}} Two rolls of a tetrahedral die
  \2 $X = First roll$, $Y = Second roll$, $Z = min(X,Y)$
  \2 $P(X=1) = 4/16 = 1/4$, $P(Z=2)=5/16$
\end{outline}

%-------------------------------------------------------------------------------------------------------
\subsection{Discrete uniform law}
\begin{outline}
\1 Assume $\Omega$ consists of $n$ equally likely elements
\1 Assume $A$ consists of $k$ elements
\1 $P(A) = k \cdot \frac{1}{n}$
\end{outline}

%-------------------------------------------------------------------------------------------------------
\subsection{Uniform probability law}
\begin{outline}
\1 $Probabiliy = Area$
\end{outline}

%-------------------------------------------------------------------------------------------------------
\subsection{Probability calculation steps}
\begin{outline}
\1 Specify the sample space
\1 Specify a probability law
\1 Identify an event of interest
\1 Calculate ...
\end{outline}

%-------------------------------------------------------------------------------------------------------
\subsection{Countable additivity axiom}
\begin{outline}
\1 If $A_1, A_2, A_3, ...$ is an infinite sequence of disjoint events,
  \2 Then $P(A_1 \cup A_2 \cup A_3 \cup ...) = P(A_1) + P(A_2) + P(A_3) + ...$
\end{outline}

%-------------------------------------------------------------------------------------------------------
\subsection{Interpretations of probabilities}
\begin{outline}
\1 A narrow view: a branch of math
  \2 Axioms $\Rightarrow$ theorems
\1 Are probabilities frequencies?
  \2 $P$(coin toss yields heads) $= 1/2$
  \2 $P$(the president of ... will be reelected) $= 0.7$
\1 Probabilities are often interpreted as:
  \2 Description of beliefs
  \2 Betting preferences
\end{outline}

%-------------------------------------------------------------------------------------------------------
\subsection{The role of probability theory}
\begin{outline}
\1 A framework for analyzing phenomena with uncertain outcomes
  \2 Rules for consistent reasoning
  \2 Used for predictions and decisions
\1 Diagram
  \2 \textbf{Real world} $\Rightarrow$ data $\Rightarrow$ \textbf{Inference/Statistics}
  \2 \textbf{Inference/Statistics} $\Rightarrow$ Models $\Rightarrow$ \textbf{Probability theory (Analysis)}
  \2 \textbf{Probability theory (Analysis)} $\Rightarrow$ Predictions / Decisions $\Rightarrow$ \textbf{Real world}
\end{outline}

%%%%%%%%%%---------------          END          ---------------%%%%%%%%%%



